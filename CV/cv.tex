\documentclass[11pt,letterpaper,colorlinks,linkcolor=true,sans]{moderncv}

\usepackage[scale=0.85]{geometry}
\usepackage{multicol}

\moderncvtheme{banking}
\moderncvcolor{black}

\firstname{Akash}
\familyname{Dhiraj}
%\homepage{akashdhiraj.com}
\social[github]{polybeandip}
\email{akashdhiraj2019@gmail.com}
\nopagenumbers{}

\newcommand*{\cvitemwithcommentnocolon}[4][.25em]{%
  \savebox{\cvitemwithcommentmainbox}{\ifthenelse{\equal{#2}{}}{}{\hintstyle{#2} }#3}%
  \setlength{\cvitemwithcommentmainlength}{\widthof{\usebox{\cvitemwithcommentmainbox}}}%
  \setlength{\cvitemwithcommentcommentlength}{\maincolumnwidth-\separatorcolumnwidth-\cvitemwithcommentmainlength}%
  \begin{minipage}[t]{\cvitemwithcommentmainlength}\usebox{\cvitemwithcommentmainbox}\end{minipage}%
  \hfill% fill of \separatorcolumnwidth
  \begin{minipage}[t]{\cvitemwithcommentcommentlength}\raggedleft\small\itshape#4\end{minipage}%
  \par\addvspace{#1}
}

\newcommand{\entry}[6]{\cventry[0.85em]{#4}{#3}{#2}{#1}{#5}{#6}}
\newcommand{\subtext}[1]{\bgroup\small\rmfamily\color{black!70}#1\egroup}
\newcommand{\job}[4]{\smallskip\cvitemwithcomment{#2}{#1}{#3}\unskip\subtext{#4}\smallskip}
\newcommand{\fakejob}[3]{\smallskip\cvitemwithcommentnocolon{#1}{}{#2}\unskip\subtext{#3}}
\lfoot{\emph{Last updated \today}}

\renewcommand*{\namefont}{\fontsize{20}{40}\bfseries\upshape}
\renewcommand*{\makeheaddetailssymbol}{~~~}
\renewcommand{\labelitemi}{\small{$\bullet$}}

\begin{document}

\hypersetup{urlcolor=blue}
\makecvtitle

\vspace*{-1cm}

\section{Education}
\entry{August 2021 -- May 2025}{Cornell University}{B.A. Math and Computer Science}{}{}{}

\subsection{Technical Skills}
\begin{itemize}
    \item Languages: Java, C, Python, OCaml, Javascript/HTML/CSS, Bash
    \item Frameworks and Libraries: ReactJS, Redux, NumPy, Pandas, Scikit-Learn
    \item Tools: Git, Github, Linux (Arch)
\end{itemize}

\subsection{Selected Coursework}
\vspace{-0.4cm}
\begin{multicols}{2}
    \begin{itemize}
        \item Object Oriented Programming and Data Structures (CS 2110)
        \item Discrete Structures (CS 2800)
        \item Data Structures and Functional Programming\\ (CS 3110)
        \item Introduction to Analysis of Algorithms (CS 4820)
    \end{itemize}
\end{multicols}

\vspace*{-0.47cm}

\section{Work Experience}
\job{Cornell University CS Department}{CS 3110 (Functional Programming) TA}{August 2023 -- Present}{
    \begin{itemize}
        \item Held weekly office hours, meant for students' queries on lectures, assignments, and exams.
        \item Graded, proctored, and provided feedback on assignments and exams.
        \item Guided a team of three students in completing their final class project.
    \end{itemize}
}

\job{Cornell University Math Department}{MATH 1110 (Calculus I) Course Assistant}{August 2022 -- December 2022}{
    \begin{itemize}
        \item Graded and provided feedback on homework assignments for approximately 500 students each week.
        \item Held weekly sessions to clarify concepts and assist in solving challenging problems.
    \end{itemize}
}

\job{The Math Company}{Machine Learning Intern}{May 2022 -- June 2022}{
    \begin{itemize}
        \item Used Python libraries Pandas, Scikit-Learn, and NumPy to build various supervised machine learning models.
        \item Wrote internal tools to automate common exploratory data analysis techniques.
    \end{itemize}
}

\fakejob{Math Circle Organizer}{2019 -- 2021}{
    \begin{itemize}
        \item Organized a math circle for local middle/high school students, covering advanced topics not taught at school.
        \item Planned all sessions and developed the curriculum for the class. 
            View
            \href{https://akashdhiraj.com/mathcircle_sample/Square_Coloring_Up_To_Rotations.pdf}{sample class material}.
    \end{itemize}
}

\section{Personal Projects}

\fakejob{Where's My Class}{August 2023}{
    User-friendly interface for Cornell students to visualize class locations on a map and plot routes between them.
    \begin{itemize}
        \item UI built using \textbf{ReactJS} and application state partially managed through \textbf{Redux}.
        \item Uses the Cornell Course Roster API to fetch class data.
        \item Uses the Mapbox API to display the map and fetch route data.
        \item Class data set to update daily through \textbf{Github Actions}.
    \end{itemize}
    View \href{https://github.com/polybeandip/wheres-my-class}{repository} or \href{https://akashdhiraj.com/wheres-my-class/}{website}.
}

\fakejob{Lambda Ledge}{April 2023}{
    2D-Platformer inspired by the highly acclaimed game Celeste.
    \begin{itemize}
        \item Written entirely in \textbf{OCaml}.
        \item Uses thin bindings to \textbf{SDL} to draw on the screen.
        \item Implemented the physics for the game from scratch; no game engines used.
    \end{itemize}
    View \href{https://github.com/polybeandip/lambda-ledge}{repository}.
}

\job{Programming Contest}{Advent of Code}{December 2022}{
    Annual Christmas-themed computer programming challenges. View \href{https://github.com/polybeandip/AOC}{repository}.
}

\end{document}
